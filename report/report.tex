% !TEX program = xelatex
\documentclass{article}

% Used packages
\usepackage{geometry}
\usepackage{fancyhdr}
\usepackage{lastpage}
\usepackage[scaled]{helvet}
\usepackage[T1]{fontenc}
\usepackage{hyperref}
\usepackage{indentfirst}

% Document info
\newcommand{\doctitle}{Convolutional Neural Network for Dog Breed Classification}
\newcommand{\docsubtitle}{Udacity Machine Learning Engineer Nanodegree Capstone Project}
\newcommand{\docauthor}{Alex Augusto Costa Machado}
\newcommand{\docdate}{December 10\textsuperscript{th} 2021}

% Page and margin setup
\geometry{
    a4paper,
    left=2cm,
    right=2cm,
    top=2cm,
    bottom=2cm,
    headsep=3mm,
    headheight=1.15cm
    }

% Paragraph setup
\setlength{\parindent}{2.5em}
\renewcommand{\baselinestretch}{1.25}

% Hyperlink setup
\hypersetup{
    colorlinks=true,     
    urlcolor=blue,
    }
    
% Font setup
\renewcommand\familydefault{\sfdefault} 

% Header and footer setup
\pagestyle{fancy}
\fancyhf{}
\rhead{\large{\doctitle}}
\rfoot{\thepage\ / \pageref*{LastPage}}
\renewcommand{\footrulewidth}{0.4pt}

% First page setup
\newcommand{\docinfo}{
    \thispagestyle{empty}

    \begin{center}

        \Huge\textbf{\doctitle}
        \vspace{2mm}

        \LARGE\textbf{\docsubtitle}
        
        \vspace{3mm}
        \large
        Written by \textbf{\docauthor} on \textbf{\docdate}

    \end{center}
}

% Document
\begin{document}
    \docinfo

    \section{Definition}
    \subsection{Project Overview}

    For computer vision tasks, such as a dog breed classification, an algorithm mostly used is a Convolutional Neural Network, or CNN for short, which is a model that extracts features, like texture and edges, from spatial data.

    The history behind this algorithm started during the 1950s, but it is around the year 2012 that CNNs saw a huge surge in popularity after a CNN called AlexNet achieved state-of-the-art performance labeling pictures in the \href{https://image-net.org/}{ImageNet} challenge. Alex Krizhevsky et al. published the paper "ImageNet Classification with Deep Convolutional Neural Networks" describing the winning AlexNet model.

    \subsection{Problem Statement}

    The purpose of this project is to use a Convolutional Neural Network to classify dog breeds using images as input. For images that contain a human instead of a dog, the algorithm should display which dog breed resembles the human in the picture.

    \subsection{Metrics}

    The metric to evaluate the quality of the classifier is the accuracy of the dog breed predictions.

    \section{Analysis}

    \subsection{Data Exploration}

    There are two datasets for this project: one is a dog dataset containing 8351 images of 133 different dog breeds, and the other is a human dataset consisting of 13233 photos of 5749 different people. The dog dataset is divided in train, validation and test datasets.

    \subsection{Exploratory Visualization}

    \subsection{Algorithms and Techniques}

    Three main algorithms will be used for this project: a human face detector, a dog detector, and a dog breed classifier. More about each of these algorithms is described below.

    \begin{itemize}
        \item Human Face Detector: an OpenCV's implementation of Haar feature-based cascade classifier to detect human faces in images. The pre-trained face detector is stored as XML file on \href{https://github.com/opencv/opencv/tree/master/data/haarcascades}{GitHub}.
        
        \item Dog Detector: a pre-trained VGG-16 model with weights that have been trained on \href{https://image-net.org/}{ImageNet}, a very popular dataset used for image classification and other vision tasks.
        
        \item Dog Breed Classifier: a pre-trained VGG-11 model on which the last layer is altered to output 133 classes and trained further using the dog dataset mentioned earlier.
    \end{itemize}

    \subsection{Benchmark}

    The benchmark for the dog breed classifier will be a simple CNN model designed from scratch that will be compared to the final CNN obtained from transfer learning using a pre-trained VGG-11 model.

    \section{Methodology}

    \subsection{Data Preprocessing}

    \subsection{Implementation}

    The workflow for this project is as follows:

    \begin{itemize}
        \item Data exploration: verify the information contained in the datasets.
        \item Preprocessing: crop images to make sure all have the same size and normalize the input data.
        \item Training and testing: create algorithms to train all three models mentioned earlier.
        \item Refinement: use transfer learning to improve accuracy of dog breed classifier.
        \item Performance evaluation: observe classification accuracy as the performance metric.
    \end{itemize}

    \subsection{Refinement}

    \section{Results}

    \subsection{Model Evaluation and Validation}

    \subsection{Justification}

    \section{Conclusion}

    \subsection{Free-Form Visualization}

    \subsection{Reflection}

    \subsection{Improvement}

\end{document}